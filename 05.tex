% !TEX root=main.tex

\chapter{THE VILLA AND THE GARDEN}

\linenumbers[1]

Here is Julius' villa and garden. Julius lives in a large villa. The father, the mother and three children live in the villa. Julius and Aemilia have three children: two sons and one daughter--not two daughters.

Many servants live in the villa. Their master is Julius: he has many servants. Many maidens also live in the villa. Their mistress is Aemilia: she has many maidens.

Julius lives in his villa with the big family. The father and the mother live with Marcus, Quintus and Julia. Julius and Aemilia live in the villa with the children, the servants and the maidens.

Julius' villa is in a big garden. There are many villages with gardens in Italy. In the gardens there are roses and lilies. Julius has many roses and lilies in his garden. Julius' garden is beautiful, because there are many beautiful roses and lilies in it.

Aemilia is a beautiful woman. Syra is not a beautiful woman, and her nose is not beautiful, but it is ugly. Syra, who is a good maiden, has a big and ugly nose. Julius is Aemilia's man, a beautiful woman. Julius loves Aemilia, because she is a beautiful and good woman. Aemilia loves her man Julius and lives with him. The father and the mother love their children. Julius doesn't live alone in the villa, but with Aemilia and with the big family.

There are two doors in the villa: the big door and the small door. The villa has two doors and many windows.

In Julius' villa, the big atrium\footnote{\emph{`\=atrium':} the first main room in a Roman-style house, where callers were received -- \emph{OLD (1968).}} is with a pool\footnote{\emph{`impluuium':} the quadrangular basin in the floor of an atrium which receives the rain-water from the roof -- \emph{idem.}}. What is in the pool? There is water in it. There are no windows in the atrium.

There is also a big and beautiful peristyle\footnote{\emph{`peristylium':} an inner courtyard lined with rows of columns -- \emph{idem.}}. `Peristyle' is a Greek word. In Greek and Roman villas there are big and beautiful peristyles. Is there a pool in the peristyle? It is not in the peristyle, but in the atrium. In the peristyle there is a small garden.

In the villa there are many rooms. Quintus sleeps in a small room. Is Marcus' room large? It is also small. Julius and Aemilia sleep in the big room. Where do the servants sleep? They also sleep in rooms. Are their rooms large? They are not large, and many servants sleep in a single room. Also many maidens sleep in a single room, and they don't have big rooms.

Aemilia is in the peristyle. Is she alone? Aemilia is not alone: the children are present with her in the peristyle. Julius is absent. Aemilia is in the villa without her man Julius. Where is Julius? He is in the town Tusculum without Aemilia, but with four servants.

Aemilia is in the peristyle with Marcus, Quintus and Julia. Julia sees beautiful roses in the garden and leaves away from Aemilia. Now she is not with Aemilia. Aemilia doesn't see her. The girl is in the garden.

Aemilia orders: ``Marcus and Quintus! Call Julia!''

Marcus and Quintus call Julia: ``Julia! Come!'' but Julia doesn't hear them and doesn't come.

Julia calls the boys: ``Marcus and Quintus! Come! There are many roses here.''

The boys hear Julia, but don't leave away from Aemilia.

Quintus: ``Pick roses, Julia!''

Julia picks roses and comes from the garden with five flowers.

Julia: ``Look, mother! Look, boys! Look at my roses!'' Julia is happy, roses delect her.

Aemilia: ``Behold this beautiful girl with beautiful roses!'' Aemilias' words please Julia.

Marcus: ``Roses are beautiful; the girl without roses is not!'' Marcus' words do not please Julia!

Aemilia (angry): ``Shut up, rude boy! Julia is a beautiful girl--with and without roses.''

Julia: ``Hear, Marcus and Quintus!''

Marcus: ``Mother doesn't see your ugly nose!''

Marcus and Quintus laugh: ``Hahahae!''

Julia: ``Hear, mom: these boys are laughing at me!''

Julia cries, and with a single rose leaves away from them.

Aemilia: ``Shut up, rude boys! Julias' nose is not ugly. Leave from the peristyle! Take the remaining roses and put them in the water!''

The boys take the remaining four roses and leave with them.

Aemilia, who is now alone in the peristyle, calls the maidens: ``Delia and Syra! Come!''

Delia and Syra come from the atrium. Aemilia asks them: ``Are the boys in the atrium?''

Delia answers: ``They are in the atrium.''

Aemilia: ``What are Marcus and Julius doing?''

Delia: ``The boys are taking water from the pool...''

Syra: ``...and putting them in the water.''

Now the mistress and the maidens hear the boys from the atrium: Quintus cries and Marcus laughs.

Aemilia: ``What happens now to the boys? Go, Delia! leave and ask them!'' Delia leaves from Aemilia and Syra.

Aemilia asks Syra: ``Where is Dauvs?''

Syra responds: ``In the town with the master.''

Delia comes from the atrium and calls the mistress: ``Come, oh mistress! Come!''

Aemilia: ``What happens, Delia?''

Delia: ``Quintus is in the pool!''

Aemilia: ``In the pool? What is the boy doing in the pool?''

Delia: ``He beats the water and calls you.''

Aemilia: ``What is Marcus doing?''

Delia: ``He laughs, because Quintus is in the water!''

Aemilia: ``Oh, Marcus is a rude boy! Go ahead! Call Julius, maidens!''

Syra: ``But the master is in the town.''

Aemilia: ``Oh, Julius is again absent!''

Delia: ``Go ahead! Come, mistress, and beat Marcus!''

What does the mistress do? The mistress, angry, leaves from the peristyle with the maidens.



\section[Grammar]{LATIN GRAMMAR}
\subsection{\emph{Accusative}}
\begin{enumerate}[(A)]
  \item \smallc{Masculine}\\
      Iiulius non unum fili\emph{um}, sed duos fili\emph{\=os} habet.\\
      Singular: \emph{-us} --> \emph{-um}\\
      Plural: \emph{-i} --> \emph{-\=os}
  \item \smallc{Feminine}\\
      Iulius non duas fili\emph{\=as}, sed unam fili\emph{am} habet.\\
      Singular: \emph{-a} --> \emph{-am}\\
      Plural: \emph{-ae} --> \emph{-\=as}
  \item \smallc{Neuter}\\
      Villa non unum cubicul\emph{um}, sed multa cubicul\emph{a} habet.\\
      Singular: \emph{-um --> -um}\\
      Plural: \emph{-i --> -a}
\end{enumerate}

\subsection{\emph{Ablative}}
\begin{enumerate}[(A)]
  \item \smallc{Masculine}\\
      In hort\emph{\=o} Iulii. In hort\emph{\=is} Italiae.\\
      Singular: \emph{-us --> -\=o}\\
      Plural: \emph{-i --> -\=is}
  \item \smallc{Feminine}\\
      In vill\emph{\=a} Iulii. In vill\emph{\=is} Romanis.\\
      Singular: \emph{-a --> -\=a}\\
      Plural: \emph{-ae --> -\=is}
  \item \smallc{Neuter}\\
      In oppid\emph{\=o} Tusculo. In oppid\emph{\=is} Graecis.\\
      Singular: \emph{-um --> \=o}\\
      Plural: \emph{-a --> \=is}
\end{enumerate}

\subsection{\emph{Imperative and indicative}}
``Davum voc\emph{\=a}, serve!'' Servus Davum voca\emph{t}.\\
``Iuliam voca\emph{te}, puer\=i\footnote{Vocative plural of \emph{`puer'}, not introduced yet.}!'' Pueri Iuliam voca\emph{nt}.

\nolinenumbers

\section[Exercises]{EXERCISES}
\subsection*{EXERCISE A}

\subsection*{EXERCISE B}

\subsection*{EXERCISE C}
