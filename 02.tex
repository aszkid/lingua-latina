% !TEX root=main.tex

\chapter{THE/A ROMAN FAMILY}

\linenumbers[1]

Julius is a Roman man. Aemilia is a Roman woman. Marcus is a Roman boy. Quintus is also a Roman boy. Julia is a Roman girl.

Marcus and Quintus are not men, but boys. Julius, Medus and Davus are men. Aemilia, Delia and Syra are women. Is Julia a woman? Julia is not a woman, but a little girl.

Julius, Aemilia, Marcus, Quintus, Julia, Syra, Davus, Delia and Medus are a Roman family. Julius is the father. Aemilia is the mother. Julius is the father of Marcus and Quintus. Julius is also the father of Julia. Aemilia is the mother of Marcus, Quintus and Julia. Marcus is the son of Julius. Marcus is the son of Aemilia. Quintus is also the son of Julius and Aemilia. Julia is the daughter of Julius and Aemilia.

Who is Marcus? Marcus is a Roman boy. Who is the father of Marcus? Julius is the father of Marcus. Who is the mother of Marcus? The mother of Marcus is Aemilia. Who is Julia? Julia is a Roman girl. Who is the mother of Julia? Aemilia is the mother of Julia. The father of Julia is Julius. Julia is the daughter of Julius. Who are the sons of Julius? The sons of Julius are Marcus and Quintus. Marcus, Quintus and Julia are three children. Sons and daughters are children. Marcus, Quintus and Julia are the children of Julius and Aemilia. In the family of Julius there are three children: two sons and one daughter.

Is Medus a son of Julius? Medus is not a son of Julius, but a servant of Julius. Julius is the master of Medus. Julius is the master of a servant. Davus is also a servant. Medus and Davus are two servants. Julius is the master of Medus and Davus. Julius is the master of servants and the father of children.

Is Delia the daughter of Aemilia? Delia is not the daughter of Aemilia, Delia is the maiden of Aemilia. Aemilia is the mistress of Delia. Aemilia is the mistress of a maiden. Syra is also a maiden. Delia and Syra are two maidens. Aemilia is the mistress of maidens.

Whose servant is Davus? Davus is the servant of Julius. Whose maiden is Syra? Syra is the maiden of Aemilia.

How many children are in the family? In the family of Julius there are three children. How many sons and how many daughters? Two sons and one daughter. How many servants are in the family? There are a hundred servants in the family. There are many servants in the family of Julius, and few children. Julius is the master of many servants.

`Two' and `three' are numbers. `Hundred' is also a number. The number of servants is a hundred. The number of children is three. Hundred is a large number. Three is a small number. The number of servants is large. The number of children is small. In the family of Julius there is a large number of servants, and a small number of children.

Medus is a Greek servant. Delia is a Greek maiden. In the family of Julius there are many Greek servants and Greek maidens. Is Aemilia a Greek woman? Aemilia is not a Greek woman, but a Roman one. Julius is not a Greek man, but a Roman one.

Sparta is a Greek town. Sparta, Delphi and Tisculum are three towns: two Greek towns and one Roman town. In Greece and in Italy there is a large number of towns. In Gaul there is a large number of rivers. The rivers in Gaul are large. Are the African rivers large? In Africa one river is large: the Nile; the other African rivers are small. Are the Greek islands large? Crete and Euboea are the two large islands; the other Greek islands are small.

Who is Cornelius? Cornelius is a Roman master. Julius and Cornelius are two Roman masters. Medus is not the servant of Cornelius. Medus is the servant of Julius.

Cornelius: ``Whose servant is Medus?''

Julius: ``Medus is my servant.''

Cornelius: ``Is Davus your servant?''

Julius: ``Davus is also my servant. Medus, Davus and many others are my servants.''

Cornelius: ``Is Delia your maiden?''

Julius: ``Delia is my maiden, and Syra is also my maiden. Delia, Syra and many others are my maidens. My family is large.''

Cornelius: ``How many servants are in your family?''

Julius: ``In my family there are a hundred servants.''

Cornelius: ``How many?''

Julius: ``The number of my servants is a hundred.''

Cornelius: ``A hundred servants! The number of your servants is large''


\section[Your Latin book]{YOUR LATIN BOOK}

Here are two Lain books: an old book and a new book. {\sc lingua latina} is your first Latin book. The title of your book is `{\sc lingua latina}'. Your book is not old, it is new.

In {\sc lingua latina} there are many pages and many chapters: the first chapter, the second, the third, and the others. `imperium romanum' is the title of the first chapter. The title of the second chapter is `familia romana'. In the second chapter there are six pages. In the first page of the second chapter there are many new words: \emph{vir, femina, puer, puella, familia}, and other. The number of Latin words is large!



\section[Grammar]{LATIN GRAMMAR}

\subsection{\emph{Masculine, feminine, neuter}}

\begin{enumerate}[(A)]
  \item \emph{-s} (\emph{-r}) --> \emph{masculine}\\
    Ex; filius, dominus, puer, vir, liber...
  \item \emph{-a} --> \emph{feminine}\\
    Ex; femina, puella, filia, pagina...
  \item \emph{-um} --> \emph{neuter}\\
    Ex; oppidum, imperium, vocabulum, capitulum...
\end{enumerate}

\subsection{\emph{Genitive}}
\begin{enumerate}[(A)]
  \item Masculine\\
    Singular: \emph{-i}\\
    Plural: \emph{-orum}
  \item Feminine\\
    Singular: \emph{-ae}\\
    Plural: \emph{-arum}
  \item Neuter (= masculine)\\
    Singular: \emph{-i}\\
    Plural: \emph{-orum}
\end{enumerate}

\nolinenumbers

\section[Exercises]{EXERCISES}
\subsection*{EXERCISE A}
Marcus filius Iulii est. Iulia filia Iulii est. Iulius est vir Romanus. Aemilia femina Romana est. Iulus dominus, Aemilia domina est. Medus servus Graecus est, Delia est ancilla Graeca. Sparta oppidum Graecum est.

Iulius pater Marci est. Marcus est filius Iulii et Aemiliae. Medus servus Iulii est: Iulius est dominus servi. Iulius dominus Medi et Davi est: Iulius dominus servorum est. Numerus servorum magnus est. Delia est ancilla Aemiliae: Aemilia domina ancillae est. Aemilia domina Deliae et Syrae est: Aemilia domina ancillarum est. In familia Iulii est magnus numerus servorum et ancillarum. Aemilia mater Marci et Quinti et Iuliae est. Marcus, Quintus, Iuliaque sunt liberi Iulii et Aemiliae. Numerus liberorum est tres. Numerus servorum est centum.

In pagina prima capituli secundi multa vocabula nova sunt. Numerus capitulorum non parvus est.

\subsection*{EXERCISE B}
Marcus puer Romanus est. Iulius vir Romanus est. Aemilia est femina Romana. Iulius est pater Marci et Quinti et Iulae. In familia Iulii sunt tres liberi: duo filii et una filia. Mater liberorum est Aemilia.

Quis est Davus? Davus est servus Iulii. Iulius dominus Davi est. Quae est Syra? Syra ancilla Aemiliae est. Aemilia est domina Syrae.

Cornelius: ``Quot servi sunt in familia tua?'' Iulius: ``In familia mea sunt centum (\smallc c) servi.'' Cornelius: ``Familia tua magna est!''



\subsection*{EXERCISE C}
..
